%!TEX root = ../main.tex

\newcommand{\secpar}{\kappa}
\newcommand{\aux}{\textsf{aux}}
\newcommand{\com}{\textsf{com}}
\newcommand{\CCC}{\mathcal{C}}
\newcommand{\PPP}{\mathcal{P}}
\newcommand{\ch}{\textsf{ch}}
\newcommand{\hash}{\textsf{H}}
\newcommand{\state}{\textsf{state}}
\newcommand{\seed}{\textsf{seed}}
\newcommand{\msgs}{\textsf{msgs}}
\newcommand{\Msg}{\textsf{Msg}}
\newcommand{\Expand}{\textsf{Expand}}
\newcommand{\algcomment}[1]{{\color{Gray}{\qquad//#1}}}

 \newtcolorbox{titlebox}[5]{
    enhanced,
    colframe=black,
    colback=white,
    boxrule={#3},
    arc={#2},
    auto outer arc,%
%   breakable,
    pad at break*=0pt,
    vfill before first,
    before={\par\medskip\noindent},
    after={\par\medskip},
    top=12pt, left=4pt, enlarge top by=7pt,%enlarge bottom by=7pt,%
    title={\rule[-.3\baselineskip]{0pt}{\baselineskip}\normalsize\sffamily\bfseries #1}, 
    varwidth boxed title*=-30pt, 
    attach boxed title to top left={yshift=-10pt,xshift=10pt}, 
    coltitle=black,
    boxed title style={colback=white,boxrule={#5},arc={#4},auto outer arc}
 }

 \newenvironment{systembox}[1]
 {\vspace{\baselineskip}\begin{titlebox}{Functionality \normalfont #1}{2.5pt}{1pt}{3.5pt}{1pt}}
 {\end{titlebox}}

 \newenvironment{protocolbox}[1]
 {\begin{titlebox}{ #1}{0.5pt}{0.5pt}{1pt}{0.75pt}}
 {\end{titlebox}}

\section{Applications}
\label{sec:applications}

\subsection{A Signature Scheme}
Here we describe a signature scheme using the (2,3)-OWF.  Abstractly, a
signature scheme can be built from any OWF that has an MPC protocol to
evaluatue it, by setting the public key to $y = F(x)$ for a random secret $x$,
and then proving knowledge of $x$, using a proof system based on the
MPC-in-the-head paradigm~\cite{STOC:IKOS07}.  In addition to assuming the OWF is
secure, the only other assumption require is a secure hash function. As 
no additional number theoretic assumptions are required, these type of signatures are
often proposed as secure post-quantum schemes. 

Concretely, our design follows the Picnic signature scheme~\cite{CCS:CDGORR17},
specifically the variant instantiated with the KKW proof
system~\cite{CCS:KatKolWan18} (named Picnic2 and Picnic3).  We chose to use the
KKW, rather than ZKB++ proof system since our MPC protocol to evaluate the
(2,3)-OWF is most efficient with a pre-processing phase, and KKW generally
produces shorter signatures.  We replace the OWF in Picnic, the LowMC block cipher~\cite{EC:ARSTZ15}, 
with the (2,3)-OWF, and make the corresponding changes to the MPC protocol. 

\greg{Need some motivation here for why this is interesting. Rough ideas: 
\begin{itemize}
\item No existing
signature scheme based on the (2,3)-OWF or similar assumption.  
\item Alternative
structure of the (2,3)-OWF may be easier to analyze than LowMC, which follows a
more traditional block cipher design.  
\item The (2,3)-OWF is conceptually much simpler
requires far less precomputed constants, and has a simpler implementation. 
\item Potential advantages in implementation performance
\end{itemize} }

\paragraph{OWF Description}
Recall that the OWF is defined as $y = F(x)$ where $x \in \Z_2^{n}$ and $y\in \Z_3^{t}$, and is computed as follows: 
\begin{enumerate}
\item Compute $w = Ax \in \Z_2^{m}$, where $A$ is a public, randomly chosen matrix of full rank in $\Z_2^{m\times n}$
\item Let $z\in\Z_3^{m}$ be $w$, where entries are intepreted as values mod 3
\item Compute and output $y = Gz$, where $G \in \Z_3^{m\times t}$ is public and randomly chosen as $A$ was. 
\end{enumerate}


\paragraph{An $N$-party protocol}
There will be $N$ parties for our MPC protocol, each holding a secret share of
$x$, who jointly compute $y = F(x)$.  The protocol is $N$-private, meaning that
up to $N-1$ parties may be malicious, and the secret input remains private.
Put another way, given the views of $N-1$ parties we can simulate the remaining
party's view, to prove that the $N-1$ parties have no information about the
remaining party's share. 

The preprocessing phase is similar to that in Picnic.  Each party has a random
tape that they can use to sample a secret sharing of a uniformly random value
(e.g.,  a scalar, vector, or a matrix with terms in $\Z_2$ or $\Z_3$).  Each
party samples their share $[r]$ and the shared value is implicitly defined as
$r = \sum_{i=1}^N [r]_i$.  We use $[r]_i$ to denote party $i$'s share of $r$,
or simply $[r]$ when the context makes the party's index clear.

We must also be able to create a sharing mod 3, of a secret shared value mod 2.
Let $w'\in\Z_2$ be secret shared.  Then to establish shares of $r = w' \pmod
3$, the first $N-1$ parties sample a share $[r]$ from their random tapes. The
$N$-th party's share is chosen by the prover, so that the sum of the shares is
$r$.  We refer to the last party's share as an \emph{auxiliary value}, since
it's provided by the prover as part of pre-processing.  For efficiency, the random
tape for  party $i$ is
generated by a random seed, denoted $\seed_i$, using a PRG. The state of the first
$N-1$ parties after pre-processing is a seed value used to generate the random
tape, and for the $N$-th party the state is the seed value plus the list of
auxiliary values, denoted $\aux$. 

After pre-processing, the parties enter the \emph{online} phase of the protocol. 
The prover computes $\hat{x}= x + x'$, where $x'$ is a random value, established during
preprocessing so that each party has a share $[x']$. 
The parties can then compute the OWF using the homomorphic properties of the secret sharing, 
and a share conversion gadget (to convert shares mod 2 to mod 3, used when computing $z$)
setup during preprocessing that we describe below. 
During the online phase, parties broadcast values to all other parties and we write $\msgs_i$
to denote the broadcast messages of party $i$. 

\subsubsection{MPC protocol details}

\paragraph{Preprocessing phase} Preprocessing establishes random seeds of all parties and shares of 
\begin{enumerate}
\item $x'$: a random vector in $\Z_2^{n}$,\algcomment{Sampled from random tapes}
\item $w'$: a random vector in $\Z_2^{m}$, \algcomment{Sampled from random tapes}
\item $r$: a sharing of $w' \mod 3$, shares in $\Z_3^m$,  \algcomment{Tapes + one $\aux$ value}
\item $\overline{r}$: a sharing of $1-w' \mod 3$, shares in $\Z_3^m$. \algcomment{Computed from shares of $r$}
\end{enumerate}
The shares of $\overline{r}$ are computed from shares of $r$ as follows (all arithmetic in $\Z_3^m$): the first
party computes $[\overline{r}] = 1 - [r]$, then the remaining parties compute
$[\overline{r}] = -[r]$.  Then observe that 
\[\sum_{i=1}^N [\overline{r}]_i = 1-[r]_1 - \ldots - [r]_N = 1 - \sum_{i=1}^{N}[r]_i = 1-r \]
 as required. 

\paragraph{Online phase}
The public input to the online phase is $\hat{x} = x + x'$. 
\begin{enumerate}

\item The parties locally compute $[u] = A[x'] - [w']$ and broadcast it, 
then sum the received values to get $u = Ax' - w'$.  Then they locally compute
$\hat{w} = A\hat{x} - u = Ax + w'$, where $\hat{w} \in \Z_2^{m}$. Each party broadcasts $m$ bits in this step. 

\item Let $z$ be a vector in $\Z_3^m$ and let $z_i$ denote the $i$-th component. Each party defines 
\[
    [z_i]  = \begin{cases}
                [r_i]  & \text{if $\hat{w}_i = 0$} \algcomment{\text{Note that $[r_i] = [w_i']$}}\\
                [\overline{r}_i]  & \text{if $\hat{w}_i = 1$} \algcomment{\text{Note that $[\overline{r}_i] = [1- w_i']$}}\\
            \end{cases}
\]
then localy computes $[y] = G[z]$. All parties broadcast $[y]$ and reconstruct the output $y\in\Z_{3}^t$. 
In this step each party broadcasts $t$ values in $\Z_3$.
\end{enumerate}

\paragraph{Correctness} The protocol correctly computes the (2,3)-OWF.  The
first step computes $w = Ax$, while keeping it masked with a random $w'$,
updating the public value $\hat{x} = x + x'$ with $\hat{w} = w + w'$.  The
second step is where the bits of $w$ are cast from $\Z_2$ to $\Z_3$.  The
parties have sharings of $w'$ and $1-w'$ mod 3. Now, the key observation is
that when $\hat{w} = 0$, then $w$ and $w'$ are the same, and when $\hat{w} =
1$, $w$ and $w'$ are different. So in the first case we set the shares of $ z =
w \mod 3$ to the shares of $[w'] \mod 3$, and when $\hat{w} = 1$, we set the
shares of $z$ to the complement of $w'$.

\paragraph{Communication Costs}
Here we quantify the cost of communication for the $\aux$ values and the broadcast $\msgs$ of one party,
as this will directly contribute to the signature size in the following section. 
Let $\ell_3$ be the bitlength of an element in $\Z_3$; the direct encoding has
$\ell_3 = 2$, but with compression we can reudce $\ell_3$ to as little as
$\log_2(3) \approx 1.58$. \greg{TODO: crossref.  I'm assuming the details of
this will be somewhere in the paper? I don't know them :)}   The size of the
$\aux$ information is $m\ell_3$, the MPC input value has size $n$ bits, 
and the broadcast values have size $m + t\ell_3$ bits (per party). 
The total in bits is thus 
\begin{equation} \label{eqn:sizeMPC}
|\textsf{MPC}(n,m,t)| = m(\ell_3 + 1) + n + t\ell_3 
\end{equation}
which is $2.58m + n + 1.58t$ when $\ell_3 = 1.58$. 
For the parameters $(n,m,t) = (256, 256, 81)$ the total is 1045 bits for
$(n,m,t) = (256, 256, 160)$ the total is 1170 bits, and for
$(n,m,t)=(128, 400, 81)$ the total is 1288 bits.  This compares favorably to
Picnic at the same security level, which communicates 1032 bits for the $\aux$
and $\msgs$ when LowMC uses a full S-box layer, and 1200 bits when LowMC
uses a partial S-box layer~\cite{TCHES:KalZav20}. 

\subsubsection{Signature Scheme Details}
Given the MPC protocol above, we can compute the values $\hat{x}$, $\aux$ and
$\msgs$ for the (2,3)-OWF and neatly drop it into the KKW proof system used in
Picnic.  The signature generation and verification algorithms for the
$(2,3)$-OWF signature scheme are given in \cref{fig:23-picnic}.

\paragraph{Parameters} Let $\secpar$ be a security parameter.  The
$(2,3)$-OWF parameters are denoted $(n, m, t)$.  The KKW parameters $(N, M,
\tau)$ denote the number of parties $N$, the total number of MPC instances $M$,
and the number $\tau$ of MPC instances where the verifier checks the online
phase of simulation. 
We use a cryptographic hash function $\hash: \{0,1\}^*\to \{0,1\}^{2\secpar}$
for computing commitments, and the function $\Expand$ takes as input a random $2\secpar$-bit string
and derives a challenge having the form $(\CCC, \PPP)$ where $\CCC$ is a subset of $[M]$
of size $\tau$, and $\PPP$ is a list of length $\tau$, with entries in $[N]$. 
The challenge $(\CCC, \PPP)$ defines $\tau$ pairs $(c,p_c)$ where $c$ is the
index of  an MPC instance for which the verifier will check the online phase,
and $p_c$ is the index of the party that will remain unopened. 

\paragraph{Key generation} The signer chooses a random $x \in \Z_2^n$ as a
secret key, and a random seed $s_A \in \{0,1\}^{\secpar}$ such that $s_A$
expands to a matrix $A \in \Z_2^{m\times n}$ that is full rank (using a suitable cryptographic
function, such as the SHAKE extendable output function~\cite{sp800-185}).  We
use a unique $A$ per signer in order to avoid multi-target attacks against $F$.
Compute $y = F(x)$ and set $(y, s_A)$ as the public key.  \greg{We could
generate the public matrix $G$ from $s_A$ as well, or we could fix it once for
all signers. Is there a security reason to go with per-signer $G$? Question for
Itai.}

\paragraph{Signature generation and verification} Here we give an overview of
signature generation, for details see \cref{fig:23-picnic}. The structure of
the signature follows a three-move $\Sigma$-protocol. In the commit phase, the
prover simulates the preprocessing phase for all $M$ MPC instances, and commits
to the output (the seeds and auxiliary values).  She then simulates the online
phase for all $M$ instances and commits to the inputs, and broadcast messages.
Then a challenge is computed by hashing all commitments, together with the
message to be signed.  The challenge selects $\tau$ of the $M$ MPC instances.
The verifier will check the simulation of the online phase for these instances,
by re-computing all values as the prover did for $N-1$ of the parties, and for
remaining unopened party, the prover will provide $\msgs$ and a commitment to
the seed so that the verifier may complete the simulation and recompute all
commitments. 
For the $M-\tau$ instances not chosen by the challege, the verifier will check
the preprocessing phase only -- here the prover provides the random seeds of
all $N$ parties and the verifier can recompute the prover's commitment to the
preprocessing, in particular the verifier checks that the auxiliary values are
correct.

\paragraph{Optimizations and simplifications}
For ease of presentation, \cref{fig:23-picnic} omits some optimizations that
are essential for efficiency, but are not unique to the $(2,3)$-signature schemes, 
they are exactly as in Picnic. All random seeds in a signature are derived from
a single random root seed, using a binary tree construction. First we derive
$M$ initial seeds, once for each MPC instance, then from from the initial seed
we derive the $N$ per-party seeds. This allows the signer to reveal the seeds
of $N-1$ parties by revealing only $\log_2(N)$ intermediate seeds, similarly, 
the initial seeds for $M-\tau$ of $M$ instances may be revealed by communicating
only $(\tau)\log_2(M/\tau)$ $\secpar$-bit seeds.

For the commitments $h'^{(k)}$ to the online execution, $\tau$ are recomputed
by the verifier, and the prover provides the missing $M-\tau$.  Here we compute
the $h'^{(k)}$ as the leaves of a Merkle tree,  so that the prover can provide
the missing commitments by sending only $\tau\log_2(M/\tau)$ $2\secpar$-bit
digests. 

Finally, we omit a random salt, included in each signature, as well as counter
inputs to the hash functions to prevent multi-target
attacks~\cite{EC:DinNad19}. Also, hashing the public key when computing the
challenge, and prefixing the inputs to $\hash$ in each use for domain
separation should also be done, as in~\cite{picnic-spec}. 

\newcommand{\textunderbrace}[2]{{%
  \underbrace{#1}_{\text{#2}}
}}

\paragraph{Parameter selection and signature size}
The size of the signature in bits is:
\begin{align*}
\textunderbrace{\secpar\tau\log_2\left(\frac{M}{\tau}\right)}{initial seeds} + 
\textunderbrace{2\secpar\tau\log_2\left(\frac{M}{\tau}\right)}{Merkle tree commitments} + 
\tau\left( 
     \textunderbrace{\secpar\log_2N}{per-party seeds}  + 
     \textunderbrace{|\textsf{MPC}(n,m,t)|}{one MPC instance, \cref{eqn:sizeMPC}} 
\right)
\end{align*} 
and we note that the direct contribution of OWF choice is limited to
$|\textsf{MPC}(n,m,t)|$\footnote{The size $|\textsf{MPC}(n,m,t)|$ is a slight
overestimate since for $1/N$ instances we don't have to send $\aux$, if the
last party is unopened. In \cref{table:sig-sizes} our estimates include this,
but it's a very small difference as $\tau$ is quite small. }.  However, the
size of this term can impact the choice of $(N,M,\tau)$.  The Picnic parameters
$(N, M, \tau)$ must be chosen so that the soundness error, 
\begin{equation*} \label{eqn:soundness}
    \epsilon(N,M,\tau) = \max_{M-\tau \le k \le M} \left\lbrace  \frac{\binom{k}{M-\tau} }{\binom{M}{M-\tau} N^{k-M+\tau} } \right\rbrace\,.
\end{equation*}
is less than $2^{-\secpar}$.
There are two time/size tradeoffs, both exponential (i.e., a small decrease in
size costs a large increase in time).  The first is increasing $N$, which
reduces $\tau$ slightly, but requires a large amount of computation to
implement the MPC simulation for all parties (in particular the hashing
required to derive seeds, compute commitments and compute random tapes is the
most expensive, and independent of the OWF).  So we fix $N = 16$. 
Next we can reduce $\tau$, but only by increasing $M$ significantly,
intuitively, when there are fewer only instances checked by the verifier, we
need greater assurance in each one, and so we must audit more preprocessing
instances. 

By searching the parameter space for fixed $N$ and various options for $M,
\tau$, we get a curve, and choose from the combinations in the ``sweet spot'',
near the bend of the curve with moderate computation costs.  \cref{table:sig-sizes}
gives some options. \greg{TODO: highlight one row as being the one we'd choose}

\begin{table} \label{table:sig-sizes}
\begin{tabular}{l|l|l}
OWF Params      & KKW params        & signature size (KB)\\
$(n,m,t)$       & $(N, M, \tau)$    &   \\\hline
$(128,400,81)$  & $(16, 150, 51)$   & 15.06 \\
                & $(16, 168, 45)$   & 14.03 \\ 
                & $(16, 218, 38)$   & 12.99 \\ 
                & $(16, 250, 36)$   & 12.78 \\
                & $(16, 303, 34)$   & 12.66 \\ 
                & $(16, 352, 33)$   & 12.70 \\
Picnic3-L1      & $(16, 250, 36)$   & 12.60 \\ \hline
$(128,400,81)$  & $(64, 149, 46)$   & 15.54 \\
                & $(64, 209, 34)$   & 13.00 \\ 
                & $(64, 246, 31)$   & 12.41 \\
                & $(64, 343, 27)$   & 11.69 \\
Picnic2-L1      & $(64, 343, 27)$   & 12.36 \\
\end{tabular}
\end{table}


\paragraph{Random notes} \greg{These should be worked into the text as appropriate}
\begin{itemize}
\item OWF vs. PRF: for the OWF I don't think there is any performance advantage to
using compressed matrices (circulant matrices). So better to use
fully random ones. 

\item The parameters above need review.  I'm pretty sure $(128, 400, 81)$ is supported
by the analysis in Itai's document but the others may not be. 

\item Optionally we may compare the NIST L5 level (AES-256 equivalent, i.e., 256-bit
classical, 128-bit quantum secure), in case the security of the (2,3)-OWF
scales better than LowMC. 
\end{itemize}


\begin{figure}[p]
 \begin{minipage}[t]{1.1\textwidth}
 \begin{protocolbox}{(2,3)-OWF Signatures}
 \begin{description}
    \item[Inputs] Both signer and verifier have $F$, $y = F(x)$, the
        message to be signed $\Msg$, and the signer has the secret key $x$.  The
            parameters of the protocol $(M, N, \tau)$ are described in the text.
    \item[Commit] For each MPC instance $k\in[M]$, the signer does the following.
    \begin{enumerate}
        \item Choose uniform $\seed^{(k)}$ and use  to generate values $(\seed^{(k)}_i)_{i \in [N]}$, and compute  
        $\aux^{(k)}$ as described in the text. 
        For $i=1,\ldots N-1$, let $\state^{(k)}_i = \seed^{(k)}_i$ and  let $\state^{(k)}_{N} = \seed^{(k)}_{N} || \aux^{(k)}$.
        \item Commit to the preprocessing phase:
        \begin{align*}
        \com^{(k)}_{i} = \hash(\state^{(k)}_i) \text{ for all } i\in [N], \quad 
        h^{(k)} = \hash(\com^{(k)}_{1},\ldots,\com^{(k)}_{N}).
        \end{align*}						
        \item Compute MPC input $\hat{x}^{(k)} = x + x'^{(k)}$ based on the secret key $x$ and the random values $x'^(k)$ defined by preprocessing.
        \item Simulate the online phase of the MPC protocol, producing $(\msgs^{(k)}_i)_{i \in [N]}$.			
        \item Commit to the online phase:
        $
         h'^{(k)}= \hash(\hat{x}^{(k)}, \msgs_{1}^{(k)}, \ldots, \msgs_{N}^{(k)} ).
        $
    \end{enumerate}
    
    \item[Challenge] 
    The signer computes $\ch = \hash(h_1, \ldots h_M, h_1', \ldots, h_M',
    \Msg)$, then expands $\ch$ to the challenge $(\CCC, \PPP) := \Expand(\ch)$, as described in the text. 
    
    \item[Signature output]
    The signature $\sigma$ on $\Msg$ is 
    \[
    \sigma = (\ch, 
              ((\seed^{(k)}, h^{(k)} )_{k\not\in\CCC}, 
              ( \com^{(k)}_{p_k}, (\state^{(k)}_{i})_{i\neq p_k}, \hat{x}^{(k)}, \msgs_{p_k}^{(k)} )_{k\in\CCC})_{k\in[M]})
    \]
    
    \item[Verification] The verifier parses $\sigma$ as above, and does the following.  
    \begin{enumerate}
        \item Check the preprocessing phase. For each $k\in[M]$:
        \begin{enumerate}
        \item If $k \in \CCC$: for all $i\in[N]$  such that $i \neq p_k$, the verifier uses $\state^{(k)}_{i}$ to compute $\com^{(k)}_{i}$ as 
            the signer did, then computes $h'^{(k)} = \hash(\com^{(k)}_{1},\ldots,\com^{(k)}_{N})$ using 
            the value $\com^{(k)}_{p_k}$ from $\sigma$. 
        \item If $k\not\in\CCC$: the verifier uses $\seed^{(k)}$ to compute $h'^{(k)}$ as the signer did.
        \end{enumerate} 
        
        \item Check the online phase:
        \begin{enumerate}
            \item For each $k \in \CCC$ the verifier simulates the online phase using $(\state^{(k)}_{i})_{i \neq p_k}$,  
                masked witness $\hat{x}$ and $\msgs^{(k)}_{p_k}$ to compute $(\msgs_i)_{i \neq p_k}$. 
                Then compute $h^{(k)}$ as the signer did. The verifier outputs `invalid' if the output of the MPC simulation is not equal to $y$.
        \end{enumerate}
    \item The verifier computes $\ch' = \hash(h_1, \ldots h_M, h_1', \ldots, h_M', \Msg)$ and outputs `valid' if $\ch' = \ch$ and `invalid' otherwise. 
    \end{enumerate}
 \end{description}
 \end{protocolbox}
 \end{minipage}
	\vspace*{-10pt}
	\caption{\label{fig:23-picnic}Picnic-like signature scheme using the (2,3)-OWF and the KKW proof sytem.} 
\end{figure}


\subsection{Distributed Picnic}
\greg{Still no good ideas on how to do this}
