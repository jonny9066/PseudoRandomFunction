%!TEX root = ../main.tex

\section{Preliminaries}
\label{sec:preliminaries}

We start with some basic notation. For a positive integer $k$, we use $[k]$ to denote the set $\{1,\dots, k\}$. $\Z_p$ denotes the ring of integers modulo $p$. 




% The operator $\oplus$ denotes addition modulo 2, and the operator $\boxplus$ denotes addition modulo 3. 

We begin by defining some basic notations that we will use throughout this work. We use uppercase letters (e.g.,
A, B) to denote matrices and bold letters to denote vectors \textbf{a} . As was specified in the Boneh et al work \cite{boneh2018-darkmatter}, the input variables to the PRF functions are the key ${K}$ which is of size $m \times n$ and each input vector $\textbf{i}$ of size $n$. 
To save on bandwidth in the distributed implementation, the key was implemented as a Toeplitz matrix, requiring $m + n - 1$ bits.
At the end of the algorithm, a randomization matrix $R$ is used which is of size $t \times m = 81 \times 256$, resulting in entropy of 128 bits (as $81 = 128 \times \log_2 3$).
