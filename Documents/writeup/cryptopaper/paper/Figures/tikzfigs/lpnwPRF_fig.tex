%!TEX root = ../../main.tex

\scalebox{0.7}{
\begin{tikzpicture}
    \begin{scope}
    %  \node [draw,trapezium,minimum height=1.48cm,trapezium left angle=75,trapezium right angle=75, rotate=90,twostyle] {\rotatebox{-90}{$\mat{A}$}};
    % \node[shift={(-1,0)}] {$x$};
    \draw[twostyle] (-0.74,0.65) -- (0.74,1.05) -- (0.74,-1.05) -- (-0.74,-0.65) -- (-0.74,-0.55) -- (0.5,-0.55) -- (0.5,0.55) -- (-0.74,0.55) -- cycle;
    \draw (-0.8,0.65) node[left] {$\mat{K}$};
    \draw (-0.8,-0.65) node[left] {$x$};
    \end{scope}

    \begin{scope}[shift={(-0.9,-1.9)}]
    \tikzmath{\sidelen=0.6;\maxnum=0;}
    \foreach \num in {0,...,\maxnum} {
        \tikzmath{\yshift=(\num-(\maxnum)/2)*(\sidelen + 0.3);}
        \node [draw,regular polygon, regular polygon sides=3,scale=0.5,shift={(0,\yshift)},rotate=90] {};
    }
    \node[shift={(-0.4,0)}] {$\mat{K}$};
    \end{scope}

    \begin{scope}[shift={(-0.9,-3.1)}]
    \tikzmath{\sidelen=0.6;\maxnum=0;}
    \foreach \num in {0,...,\maxnum} {
        \tikzmath{\yshift=(\num-(\maxnum)/2)*(\sidelen + 0.3);}
        \node [draw,regular polygon, regular polygon sides=3,scale=0.5,shift={(0,\yshift)},rotate=90] {};
    }
    \node[shift={(-0.4,0)}] {$x$};
    \end{scope}


    \begin{scope}[shift={(0,-2.5)}]
        % \node [draw,trapezium,minimum height=1.48cm,trapezium left angle=75,trapezium right angle=75, rotate=90,threestyle] {\rotatebox{-90}{$\mat{A}$}};
        \draw[threestyle] (-0.74,0.65) -- (0.74,1.05) -- (0.74,-1.05) -- (-0.74,-0.65) -- (-0.74,-0.55) -- (0.5,-0.55) -- (0.5,0.55) -- (-0.74,0.55) -- cycle;
    \end{scope}

    \begin{scope}[shift={(0.9,-2.5)}]
    \tikzmath{\sidelen=0.6;\maxnum=4;}
    \foreach \num in {0,...,\maxnum} {
        \tikzmath{\yshift=(\num-(\maxnum)/2)*(\sidelen + 0.3);}
        \node [draw,regular polygon, regular polygon sides=3,scale=0.5,shift={(0,\yshift)},rotate=-90] {};
    }
    \end{scope}

    \begin{scope}[shift={(2.4,-1.25)}]
        \node[draw,circle, minimum size=1cm, twostyle] {};
        \draw (0,0.5) -- (0,-0.5);
        \draw (0.5,0) -- (-0.5,0);
        % \node[shift={(0.55,-0.55)}] {$2$};        

        \draw[->, -latex] (-1.6,0.4) -- (-0.37,0.4);

        \draw[->, -latex] (-1.35,-0.4) -- (-0.37,-0.4);


        \draw[->, -latex] (0.5,0) -- (1.5,0);

    \end{scope}



    % \begin{scope}[shift={(0.95,0)}]
    % \tikzmath{\sidelen=0.6;\maxnum=4;}
    % \foreach \num in {0,...,\maxnum} {
    %     \tikzmath{\yshift=(\num-(\maxnum)/2)*(\sidelen + 0.3);}
    %     \node [draw,regular polygon, regular polygon sides=3,scale=0.5,shift={(0,\yshift)},rotate=90] {};
    % }
    % \end{scope}
    \begin{scope}[shift={(4.65,-1)}]
    \node [draw,trapezium,minimum height=1.48cm,trapezium left angle=75,trapezium right angle=75, rotate=-90,twostyle] {\rotatebox{90}{$\mat{B}$}};
    \draw (0.75,0) node[right] {$y$};
    \end{scope}
\end{tikzpicture}
}





% \scalebox{0.8}{
% \begin{tikzpicture}
%     % \begin{scope}[scale=1.6, shift={(-0.3,0)}]
%     %     % \draw[twostyle] (-0.74,0.65) -- (0.74,1.05) -- (0.74,-1.05) -- (-0.74,-0.65) -- (-0.74,-0.55) -- (0.5,-0.55) -- (0.5,0.55) -- (-0.74,0.55) -- cycle;
%     %     %  \draw (-0.8,0.65) node[left] {$\mat{K}$};
%     %     %  \draw (-0.8,-0.65) node[left] {$x$};
%     %     \draw[twostyle] (-0.74,1.05) -- (0.74,0.65) -- (0.74,-0.65) -- (-0.74,-1.05) -- (-0.74,-0.55) -- (0.5,-0.55) -- (0.5,0.55) -- (-0.74,0.55) -- cycle;
%     %     \draw (-0.8,0.8) node[left] {$\mat{K}$};
%     %     \draw (-0.8,-0.8) node[left] {$x$};

%     % \end{scope}
%      \begin{scope}
%         \draw[twostyle] (-0.74,0.65) -- (0.74,1.05) -- (0.74,-1.05) -- (-0.74,-0.65) -- (-0.74,-0.55) -- (0.5,-0.55) -- (0.5,0.55) -- (-0.74,0.55) -- cycle;
%          \draw (-0.8,0.65) node[left] {$\mat{K}$};
%          \draw (-0.8,-0.65) node[left] {$x$};
%         % \draw[twostyle] (-0.74,1.05) -- (0.74,0.65) -- (0.74,-0.65) -- (-0.74,-1.05) -- (-0.74,-0.55) -- (0.5,-0.55) -- (0.5,0.55) -- (-0.74,0.55) -- cycle;
%         % \draw (-0.8,0.8) node[left] {$\mat{K}$};
%         % \draw (-0.8,-0.8) node[left] {$x$};
%     \end{scope}
%     \begin{scope}[shift={(0.95,0)}]
%     \tikzmath{\sidelen=0.6;\maxnum=4;}
%     \foreach \num in {0,...,\maxnum} {
%         \tikzmath{\yshift=(\num-(\maxnum)/2)*(\sidelen + 0.3);}
%         \node [draw,regular polygon, regular polygon sides=3,scale=0.5,shift={(0,\yshift)},rotate=90] {};
%     }
%     \end{scope}
%     \begin{scope}[shift={(1.85,0)}]
%     \node [draw,trapezium,minimum height=1.48cm,trapezium left angle=75,trapezium right angle=75, rotate=-90,threestyle] {\rotatebox{90}{$\mat{B}$}};
%     \draw (0.75,0) node[right] {$y$};
%     \end{scope}
% \end{tikzpicture}
% }
